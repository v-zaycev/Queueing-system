\documentclass[specialist,
substylefile = spbu_report.rtx,
subf,href,colorlinks=true, 12pt]{disser}

\usepackage[a4paper,
mag=1000, includefoot,
left=1.5cm, right=1.5cm, top=2cm, bottom=2cm, headsep=1cm, footskip=1cm]{geometry}

\usepackage{color}
\definecolor{black}{RGB}{0,0,0}

%\usepackage{hyperref}
\usepackage{mathtext}
\usepackage{outlines}
\usepackage{amsmath,amssymb,amsthm,amscd,amsfonts}
\usepackage{cmap}
\usepackage[utf8]{inputenc}
\usepackage[T2A]{fontenc}
\usepackage[russian]{babel}
\usepackage{euscript}
\usepackage{mathdots}
\usepackage{graphicx}
%\usepackage[colorlinks,citecolor=black, linkcolor=black]{hyperref}
\usepackage[russian]{cleveref}
\usepackage{epstopdf}
\usepackage{longtable} 
\usepackage{multirow}
\usepackage{float}
\usepackage{wrapfig}
\usepackage{caption}
\usepackage{esdiff}
\usepackage{mathtools, bigstrut}
\usepackage[shortlabels]{enumitem}
\usepackage{listings}
\usepackage{setspace}
\usepackage{xcolor}
\usepackage{enumitem}

\usepackage{epstopdf}
\usepackage{array}


\newtheorem*{theorem}{Теорема}
\newtheorem*{Corollary}{Следствие}
\newtheorem*{Remark}{Замечание}
\newtheorem*{Lemma}{Лемма}
\newtheorem*{Definition}{Определение}
\newcommand*\rfrac[2]{{}^{#1}\!/_{#2}}
%\newcommand{\tg}[1]{\mathrm{tg} \left(#1\right)}
\sectionfont{\fontsize{18}{15}\selectfont}
\subsectionfont{\fontsize{16}{13}\selectfont}
\subsubsectionfont{\fontsize{14}{11}\selectfont}
%\DeclareMathOperator{\deter}{det}
\DeclareMathOperator{\rk}{rk}
\DeclareMathOperator{\re}{Re}
\DeclareMathOperator{\im}{Im}
%\DeclareMathOperator{\arctg}{arctg}


\definecolor{codegreen}{rgb}{0,0.6,0}
\definecolor{codegray}{rgb}{0.5,0.5,0.5}
\definecolor{codepurple}{rgb}{0.58,0,0.82}
\definecolor{backcolour}{rgb}{0.95,0.95,0.92}

%Code listing style named "mystyle"
\lstdefinestyle{mystyle}{
	backgroundcolor=\color{backcolour}, commentstyle=\color{codegreen},
	keywordstyle=\color{red},
	numberstyle=\tiny\color{codegray},
	stringstyle=\color{blue},
	basicstyle=\ttfamily\footnotesize,
	breakatwhitespace=false,         
	breaklines=true,                 
	captionpos=b,                    
	keepspaces=true,                 
	numbers=left,                    
	numbersep=5pt,                  
	showspaces=false,                
	showstringspaces=false,
	showtabs=false,                  
	tabsize=2
}

%"mystyle" code listing set
\lstset{style=mystyle}


\AtBeginDocument{\addtocontents{toc}{\protect\thispagestyle{plain}}} 
\setcounter{tocdepth}{2}
%\linespread{1} 
\begin{document}
	
	%
	% Титульный лист на русском языке
	%
	% Название организации
	\institution{%		
		\begin{spacing}{1}
			Министерство науки и высшего образования Российской Федерации\\
			\textbf{--------------------------------}\\
			Санкт-Петербургский политехнический университет Петра Великого\\
			Высшая школа программной инженерии
		\end{spacing}
	}
	
	\title{Отчёт по курсовой работе}
	
	% Тема
	\topic{\textbf{"Методы имитационного моделирования"}}
	
	% Автор
	\author{Зайцев В.А.}
	\group{студент гр. 5130904/30103}
	
	% Научный руководитель
	\sa       {Дробинцев Д.Ф.}
	%\sastatus {проф, д.т.н. }
	
	% Город и год
	\city{Санкт-Петербург}
	\date{\number\year}
	
	\maketitle
	\pagestyle{plain}
	\newpage
	\tableofcontents
	
	\section*{\centering Постановка задачи}
	\addcontentsline{toc}{section}{Постановка задачи}
	
	Целью курсовой работы является создание модели вычислительной системы (ВС) или ее части на некотором уровне детализации, описывающей и имитирующей ее структуру и функциональность.
	
	Каждый реальный объект (реальная ВС) обладает бесконечной сложностью, множеством характеристик, внутренних и внешних связей. Модель есть приближенное описание объекта с целью получения требуемых результатов с определенной точностью и достоверностью.
	
	При необходимости исследования поведенческих характеристик ВС в процессе исследования выгодно использовать не сам объект, а его модель. Степень приближения модели к описываемому объекту может быть различной и зависит от требований задачи.
	
	Существуют различные типы моделей:
	\begin{itemize}
		\item Аналитические (математические) модели
		\item Аналоговые модели
		\item Физические модели
		\item Имитационные модели
	\end{itemize}

	Последний тип моделей является предметом нашего изучения.
	
	Одним из подходов к построению имитационной модели является построение ее в виде системы массового обслуживания (СМО), с характерной для СМО терминологией: источник, буфер, прибор, диспетчер, заявка (требование).
	\newpage
	
	\section*{\centering Исходные данные}
	\addcontentsline{toc}{section}{Исходные данные}

	\subsection*{\centering Формализованная схема и описание СМО}
	\addcontentsline{toc}{subsection}{Формализованная схема и описание СМО}

	\begin{center}	
		\includegraphics[scale=0.5]{images/scheme.png}
	\end{center}

	\setlength\extrarowheight{3pt}
	\begin{table}[H]
		\centering
		\begin{tabular}{|l|l|l|l|l|l|l|l|l|l|}
			\hline
			11&ИБ&ИЗ1&ПЗ2&Д10З3&Д10О3&Д2П1&Д2Б2&ОР1&ОД2\\ \hline
		\end{tabular}
	\end{table}
	
	\begin{itemize}
		\item Параметры элементов модели.
	  	\begin{itemize}
		  	\item Источники: 
		  	\begin{itemize}
				\item ИБ~--- бесконечный источник;
				\item ИЗ1~--- пуассоновский закон распределения заявок;
		  	\end{itemize}
			\item Приборы: 	 
			\begin{itemize}
				\item ПЗ2~--- равномерный закон распределения времени обслуживания;
			\end{itemize}
		\end{itemize} 
		\item Описание дисциплин постановки и выбора.
		\begin{itemize}
			\item Буферизации: 
			\begin{itemize}
				\item Д1ОЗ3~--- постановка заявки в очередь на первое от начала свободное место;
			\end{itemize}
			\item Отказа: 	 
			\begin{itemize}
				\item Д1ОО3~--- заявка, раньше других вставшая в буфер получает отказ, уходит из системы и на её место встает пришедшая заявка;
			\end{itemize}
			\item Выбора заявки на обслуживание:
			\begin{itemize}
				\item Д2Б1~--- LIFO (последним пришел — первым обслужен);
			\end{itemize}
			\item Выбора обслуживающего прибора:
			\begin{itemize}
				\item Д2П1~--- приоритет по номеру прибора;
			\end{itemize}
		\end{itemize} 
		\item Виды отображения результатов работы программной модели.
		\begin{itemize}
			\item Динамическое отражение результатов (пошаговый режим): 
			\begin{itemize}
				\item ОД2~--- формализованная схема модели, текущее состояние;
			\end{itemize}
			\item Отражение	результатов	после сбора статистики (автоматический режим): 	 
			\begin{itemize}
				\item ОР1~--- сводная таблица результатов;
			\end{itemize}
		\end{itemize} 
	\end{itemize}
	\newpage
	
	
	\section*{\centering Пример временной диаграммы функционирования системы}
	\addcontentsline{toc}{section}{Пример временной диаграммы функционирования системы}


	\begin{center}
		\includegraphics[width=\textwidth]{images/waveform2.png}
	\end{center}
	\newpage
	
	\section*{\centering Обобщённая блок-схема}
	\addcontentsline{toc}{section}{Обобщённая блок-схема}

	
	\begin{center}
		\includegraphics[width=\textwidth]{images/common_scheme.png}
	\end{center}
	\newpage
	
	
	\section*{\centering Пример технической системы,\\ удовлетворяющей формализованному описанию}
	\addcontentsline{toc}{section}{Пример технической системы, \\ удовлетворяющей формализованному описанию}

	
	\begin{table}[H]
		\centering
		\begin{tabular}{|p{5cm}|p{10cm}|}
			\hline
			Техническая система&Система видеонаблюдения\\ \hline
			Источники & Камеры видеонаблюдения, которые отправляют кадры только при обнаружении движения в кадре. Количество камер может варьироваться от 20 до 50 устройств.\\ \hline
			Приборы & Приборами являются видеокарты, которые анализируют кадры (поиск лиц, трекинг объектов между несколькими камерами).\\ \hline
			Буфер&Выделенная область оперативной памяти, в которую записываются поступающие кадры.\\ \hline
			Дисциплина постановки в	буфер&Постановка на первое свободное место в буфере\\ \hline
			Дисциплина выбора из буфера&LIFO - последние кадры обрабатываются первыми, так как они наиболее актуальны для текущего состояния системы.	\\ \hline
			Дисциплина отказа&Самые старые кадры в видеопамяти перезаписываются при нехватке места.\\ \hline
			Дисциплина постановки на обслуживание&Выбирается первая свободная видеокарта.\\ \hline
		\end{tabular}
	\end{table}
	
	\subsection*{\centering Ограничения и требуемые характеристики}
	\addcontentsline{toc}{subsection}{Ограничения и требуемые характеристики}
	
	\begin{table}[H]
		\centering
		\begin{tabular}{|p{7cm}|p{5cm}|}
			\hline
			Количество камер & От 20 до 50\\ \hline
			Размер заявки & 1Мб\\ \hline
			Размер буфера & от 25Мб до 50Мб\\ \hline
			Количество видеокарт & от 1 до 10\\ \hline
			Среднее время между кадрами& $\lambda=1$с\\ \hline
			Время обработки кадров& 350-500 мс\\
			&150-220 мс\\
			&50-90 мс\\
			&40-60 мс	\\ \hline
		\end{tabular}
	\end{table}
	
	Вероятность отказа (пропуска кадров) должна составлять не более 10\%.\\ 
	\indent Загрузка приборов более 90\% при максимально возможном числе камер.
	
	\begin{table}[H]
		\centering
		\caption*{Стоимость компонентов системы}
		\begin{tabular}{|p{6cm}|p{7cm}|p{3cm}|}
			\hline
			Наименование & Время обслуживания заявки & Цена\\ \hline
			NVIDIA GeForce GTX 1650 & 350-500 мс & 18 000 р.\\ \hline
			NVIDIA GeForce RTX 3050 & 150-220 мс & 40 000 р.\\ \hline
			NVIDIA GeForce RTX 4080 & 50-90 мс & 120 000 р.\\ \hline
			NVIDIA RTX A4000 & 40-60 мс & 150 000 р.\\ \hline
		\end{tabular}
	\end{table}
	
	Поскольку число камер и размер кадров таковы, что память буфера значительно меньше оперативной памяти стандартного компьютера (современный ПК имеет от 4 Гб ОЗУ) на несколько порядков, то можем считать, что дополнительных затрат на память не будет. 
	
	Нам требуется найти самую дешёвую конфигурацию системы, которая будет удовлетворять всем требованиям при максимальной нагрузке т.е. при максимальном числе подключенных камер равном 50. Мы можем варьировать размер буфера и число камер, а также менять различные виды видеокарт. 	\\
	\newpage
	
	
	\section*{\centering Модульная структура}
	\addcontentsline{toc}{section}{Модульная структура}

	Разработка производилась в среде VisualStudio на языке С++ и среде VS Code на языке Python с использованием графической библиотеки Qt.
	Приложение является объектно-ориентированным и содержит следующие классы: 
	\begin{itemize}[noitemsep]
		\item Request~--- класс описывающий т пакет данных от датчика транспортного потока 
		\item Event~--- класс описывающий тип события и время его наступления и заявку, с которой это событие связано
		\item RequestSource~--- класс источника, который создаёт объекты заявок Request
		\item Handler~---  прибора, который обрабатывает заявки Request
		\item Buffer~--- класс буффера, хранящий некоторое число заявок Request
		\item UniformGenerator~--- класс генератора случайных чисел с равномерным распределением и в интервале $\left[a,b\right]$
		\item PoissonGenerator~--- класс генератора случайных чисел с экспоненциальным распределением с параметром $\lambda$
		\item QueueingSystem — класс реализующий интерфейс взаимодействия источников, приборов и буфера, а также сбор статистики 
		\item Набор классов MainWindow, Tab1Widget, Tab2Widget и другие, реализующие графический интерфейс приложения
	\end{itemize}
	\newpage
	
	\section*{\centering Описание работы программы}
	\addcontentsline{toc}{section}{Описание работы программы}
	
	Графический интерфейс приложения содержит два основных окна: для пошагового моделирования и для моделирования  в автоматическом режиме.
	
	Оба окна содержат поля ввода всех основных параметров системы (число источников, число приборов, размер буфера, время моделирования, параметры времени обработки на приборе и интенсивности источников), а также набор кнопок для сохранения и сброса настроек, а также управления ходом моделирования.
	
	В нижней части окна пошагового моделирования содержатся блоки элементов системы (блок источников, блок ячеек буфера и блок приборов), которые отражают состояние системы на момент очередного события. 
	\begin{itemize}
		\item Источники содержат информацию о времени формирования следующей заявки.
		\item Ячейки буфера содержат информацию о времени формирования хранимой заявки и номере прибора, от которого эта заявка поступила. Если ячейка буфера не содержит заявку, то указывается, что ячейка пуста.
		\item Приборы содержат информацию о времени своего освобождения. Если прибор не обрабатывает заявку, то указывается, что он свободен. 
	\end{itemize}
	
	\begin{center}
		\includegraphics[width=\textwidth]{images/step0.png}
	\end{center}
	\begin{center}
		\includegraphics[width=\textwidth]{images/step1.png}
	\end{center}
	\begin{center}
		\includegraphics[width=\textwidth]{images/step2.png}
	\end{center}
	\begin{center}
		\includegraphics[width=\textwidth]{images/step3.png}
	\end{center}
	
	В нижней части окна автоматического моделирования содержатся таблицы, содержащие расширенную статистику по отдельным источникам и приборам. Также над ними отдельно показывается общая статистика по числу заявок, загруженности приборов и вероятности отказов заявок.  
	
	\begin{center}
		\includegraphics[width=\textwidth]{images/auto0.png}
	\end{center}

	\begin{center}
		\includegraphics[width=\textwidth]{images/auto4.png}
	\end{center}
	
	
	\iffalse
	\section{Пример технической системы, удовлетворяющей формализованному описанию}
	
	\begin{table}[H]
		\centering
		\begin{tabular}{|p{5cm}|p{10cm}|}
			\hline
			Техническая система&Система видеонаблюдения\\ \hline
			Источники & Камеры видеонаблюдения, которые отправляют кадры только при обнаружении движения в кадре. Количество камер может варьироваться от 20 до 50 устройств.\\ \hline
			Приборы & Приборами являются видеокарты, которые анализируют кадры (поиск лиц, трекинг объектов между несколькими камерами).\\ \hline
			Буфер&Выделенная область оперативной памяти, в которую записываются поступающие кадры.\\ \hline
			Дисциплина постановки в	буфер&Постановка на первое свободное место в буфере\\ \hline
			Дисциплина выбора из буфера&LIFO - последние кадры обрабатываются первыми, так как они наиболее актуальны для текущего состояния системы.	\\ \hline
			Дисциплина отказа&Самые старые кадры в видеопамяти перезаписываются при нехватке места.\\ \hline
			Дисциплина постановки на обслуживание&Выбирается первая свободная видеокарта.\\ \hline
		\end{tabular}
	\end{table}
	
	\section{Ограничения и требуемые характеристики}	
	
	Вероятность отказа должна составлять не более 10\%.\\ 
	\indent Загрузка приборов более 90\%
	
	\begin{table}[H]
		\centering
		\begin{tabular}{|p{7cm}|p{5cm}|}
			\hline
			Количество камер & От 20 до 50\\ \hline
			Размер заявки & 1Мб\\ \hline
			Размер буфера & от 25Мб до 50Мб\\ \hline
			Количество видеокарт & от 1 до 10\\ \hline
			Среднее время между кадрами& $\lambda=1$с\\ \hline
			Время обработки кадров& 350-500 мс\\
					  			  &150-220 мс\\
								  &50-90 мс\\
								  &40-60 мс	\\ \hline
		\end{tabular}
	\end{table}

	\fontsize{18}{15}\selectfont
	\textbf{Стоимость компонентов системы}
	\normalsize
	\begin{table}[H]
		\centering
		\begin{tabular}{|p{6cm}|p{7cm}|p{3cm}|}
			\hline
			Наименование & Время обслуживания заявки & Цена\\ \hline
			NVIDIA GeForce GTX 1650 & 350-500 мс & 18 000 р.\\ \hline
			NVIDIA GeForce RTX 3050 & 150-220 мс & 40 000 р.\\ \hline
			NVIDIA GeForce RTX 4080 & 50-90 мс & 120 000 р.\\ \hline
			NVIDIA RTX A4000 & 40-60 мс & 150 000 р.\\ \hline
		\end{tabular}
	\end{table}

	Поскольку число камер и размер кадров таковы, что память буфера значительно меньше оперативной памяти стандартного компьютера (современный ПК имеет от 4 Гб ОЗУ) более чем на порядок, то можем считать, что дополнительных затрат на память не будет. 
	
	Нам требуется найти самую дешёвую конфигурацию системы, которая будет удовлетворять всем требованиям при максимальной нагрузке т.е. при максимальном числе подключенных камер равном 50. Мы можем варьировать размер буфера и число камер, а также менять различные виды видеокарт. 	\\
	\fi
	\newpage
	\section*{\centering Результаты работы имитационной модели}
	\addcontentsline{toc}{section}{Результаты работы имитационной модели}
	\subsection*{\centering  Определение количества реализаций}
	\addcontentsline{toc}{subsection}{Определение количества реализаций}
	
	Количество реализаций, необходимое для получения нужной точности при заданной доверительной вероятности, можно оценивать по формуле:
	\begin{equation*}
		N=\frac{t_{\alpha}^2(1-p)}{p\delta^2}=2429,
	\end{equation*}
	где:
	\begin{itemize}
		\item $p$~--- вероятность отказа в обслуживании заявки;
		\item $t_{\alpha}=1.643$ для $\alpha=0.9$ (доверительная вероятность $90\%$);
		\item $\delta=0.1$~--- относительная точность $(10\%)$.
	\end{itemize}
	
	По результатам работы программы получено, что в большинстве случаев хватает около $10.000$ заявок для достижения нужной точности. Однако, в зависимости	от конфигурации, может потребоваться ощутимо большее количество заявок (до $25.000$).
	
	\begin{figure}[H]
		\centering
		\includegraphics[width=\textwidth]{images/auto5.png}
	\end{figure}

	\subsection*{\centering  Анализ результатов, выводы и рекомендации по выбору конфигурации системы.}
	\addcontentsline{toc}{subsection}{Анализ результатов, выводы и рекомендации по выбору конфигурации системы.}
	\subsubsection*{\centering NVIDIA GeForce GTX 1650}
	\addcontentsline{toc}{subsubsection}{NVIDIA GeForce GTX 1650}
	
	В случае с NVIDIA GeForce GTX 1650 мы можем отметить, что на всех конфигурациях она выдаёт максимально возможную или близкую к максимальной загрузку всех видеокарт, но при этом ни одна конфигурация не позволяет снизить вероятность отказа хотя бы до 50\% не говоря уже о целевых 10\%.
	
	\begin{figure}[H]
		\centering
		\includegraphics[width=\textwidth]{images/1650_usage.pdf}
	\end{figure}

	\begin{figure}[H]
		\centering
		\includegraphics[width=\textwidth]{images/1650_rejection_rate.pdf}
	\end{figure}

	\begin{figure}[H]
		\centering
		\includegraphics[width=\textwidth]{images/1650_total.pdf}
	\end{figure}
	
	
	\subsubsection*{\centering NVIDIA GeForce RTX 3050}
	\addcontentsline{toc}{subsubsection}{NVIDIA GeForce RTX 3050}

	Для NVIDIA GeForce RTX 3050 можно найти несколько конфигураций которые будут удовлетворять всем необходимым условиям. Первый набор конфигураций состоит из 9 видеокарт и буфера размером от 5 до 25 заявок. Второй набор состоит из 10 видеокарт буфера размером от 8 до 25 заявок. Самой оптимальной будет конфигурация из 9 видеокарт и 5 местного буфера, однако при необходимости можно будет дополнительно снизить вероятность отказа за счёт добавления ещё одной видеокарты.   
	
	\begin{figure}[H]
		\centering
		\includegraphics[width=\textwidth]{images/3050_usage.pdf}
	\end{figure}
	
	\begin{figure}[H]
		\centering
		\includegraphics[width=\textwidth]{images/3050_rejection_rate.pdf}
	\end{figure}
	
	\begin{figure}[H]
		\centering
		\includegraphics[width=\textwidth]{images/3050_total.pdf}
	\end{figure}


	\subsubsection*{\centering NVIDIA GeForce RTX 4080}
	\addcontentsline{toc}{subsubsection}{NVIDIA GeForce RTX 4080}
	
	В случае с NVIDIA GeForce RTX 4080 мы не можем найти баланс между загруженностью карт и вероятностью отказа. Если в системе меньше 4 видеокарт, то мы имеем высокую загрузку при высоком проценте отказов, а если 4 и более, то наоборот~--- загруженность становится менее $90\%$ и процент отказов становится менее $10\%$.
	
	\begin{figure}[H]
		\centering
		\includegraphics[width=\textwidth]{images/4080_usage.pdf}
	\end{figure}
	
	\begin{figure}[H]
		\centering
		\includegraphics[width=\textwidth]{images/4080_rejection_rate.pdf}
	\end{figure}
	
	\begin{figure}[H]
		\centering
		\includegraphics[width=\textwidth]{images/4080_total.pdf}
	\end{figure}
	

	\subsubsection*{\centering NVIDIA RTX A400}
	\addcontentsline{toc}{subsubsection}{NVIDIA RTX A400}
	
	Для NVIDIA RTX A400 мы также не можем найти баланс между загруженностью карт и вероятностью отказа. Но если для NVIDIA GeForce RTX 4080 переход был между 3 и 4 видеокартами в системе, то теперь он находится между 2 и 3. Можно отметить, что проблема с более мощными NVIDIA GeForce RTX 4080 и NVIDIA RTX A400 состоит в том, что они дают слишком высокий прирост общей производительности системы при добавлении одной видеокарты, из-за чего не удаётся получить сбалансированные настройки для наших ограничений.
		
	\begin{figure}[H]
		\centering
		\includegraphics[width=\textwidth]{images/A400_usage.pdf}
	\end{figure}
	
	\begin{figure}[H]
		\centering
		\includegraphics[width=\textwidth]{images/A400_rejection_rate.pdf}
	\end{figure}
	
	\begin{figure}[H]
		\centering
		\includegraphics[width=\textwidth]{images/A400_total.pdf}
	\end{figure}
	
	Таким образом мы получили несколько вариантов конфигурации удовлетворяющих ограничениям. Все подходящие конфигурации используют NVIDIA GeForce RTX 3050. Самой оптимальной из подходящих является вариант с 9 видеокартами и 5 местами в буфере, однако можно добавить ещё одну видеокарту или увеличить число мест в буфере до максимальных 25 чтобы ещё сильнее уменьшить вероятность отказа оставаясь в рамках необходимой загрузки. Остальные видеокарты в любой конфигурации не могут удовлетворить хотя бы одному параметру. 
	\newpage
	\section*{\centering Вывод}
	\addcontentsline{toc}{section}{Вывод}

	В ходе курсовой работы была написана имитационная модель системы массового обслуживания на языке C++, а также реализован графический интерфейс для неё на языке Python с использованием графической библиотеки Qt. С помощью данной программы была проанализирована реальная система видеонаблюдения и подобрана максимально выгодная комплектация системы, удовлетворяющая поставленным требованиям.
	\newpage
	\section*{\centering Список литературы}
	\addcontentsline{toc}{section}{Список литературы}
	\begin{enumerate}
		\item Гмурман В.Е. Теория вероятностей и математическая статистика : учеб. пособие для вузов В. Е. Гмурман .— 12-е изд., перераб. — М. : Юрайт, 2010 .— 478 с.
		\item Бусленко Н.П. Моделирование сложных систем. М.: Наука, 1978
		\item Александрова О.В., Котляров В.П. Архитектура вычислительных систем. Методические указания для курсового проектирования. СПбПУ, 2018г. — 73 с.
	\end{enumerate}
\end{document}