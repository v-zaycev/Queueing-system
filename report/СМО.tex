\documentclass[specialist,
substylefile = spbu_report.rtx,
subf,href,colorlinks=true, 12pt]{disser}

\usepackage[a4paper,
mag=1000, includefoot,
left=1.5cm, right=1.5cm, top=2cm, bottom=2cm, headsep=1cm, footskip=1cm]{geometry}

\usepackage{color}
\definecolor{black}{RGB}{0,0,0}

%\usepackage{hyperref}
\usepackage{mathtext}
\usepackage{outlines}
\usepackage{amsmath,amssymb,amsthm,amscd,amsfonts}
\usepackage{cmap}
\usepackage[utf8]{inputenc}
\usepackage[T2A]{fontenc}
\usepackage[russian]{babel}
\usepackage{euscript}
\usepackage{mathdots}
\usepackage{graphicx}
%\usepackage[colorlinks,citecolor=black, linkcolor=black]{hyperref}
\usepackage[russian]{cleveref}
\usepackage{epstopdf}
\usepackage{longtable} 
\usepackage{multirow}
\usepackage{float}
\usepackage{wrapfig}
\usepackage{caption}
\usepackage{esdiff}
\usepackage{mathtools, bigstrut}
\usepackage[shortlabels]{enumitem}
\usepackage{listings}
\usepackage{setspace}
\usepackage{xcolor}


\usepackage{epstopdf}
\usepackage{array}


\newtheorem*{theorem}{Теорема}
\newtheorem*{Corollary}{Следствие}
\newtheorem*{Remark}{Замечание}
\newtheorem*{Lemma}{Лемма}
\newtheorem*{Definition}{Определение}
\newcommand*\rfrac[2]{{}^{#1}\!/_{#2}}
%\newcommand{\tg}[1]{\mathrm{tg} \left(#1\right)}
\sectionfont{\fontsize{18}{15}\selectfont}
\subsectionfont{\fontsize{16}{13}\selectfont}
\subsubsectionfont{\fontsize{14}{11}\selectfont}
%\DeclareMathOperator{\deter}{det}
\DeclareMathOperator{\rk}{rk}
\DeclareMathOperator{\re}{Re}
\DeclareMathOperator{\im}{Im}
%\DeclareMathOperator{\arctg}{arctg}


\definecolor{codegreen}{rgb}{0,0.6,0}
\definecolor{codegray}{rgb}{0.5,0.5,0.5}
\definecolor{codepurple}{rgb}{0.58,0,0.82}
\definecolor{backcolour}{rgb}{0.95,0.95,0.92}

%Code listing style named "mystyle"
\lstdefinestyle{mystyle}{
	backgroundcolor=\color{backcolour}, commentstyle=\color{codegreen},
	keywordstyle=\color{red},
	numberstyle=\tiny\color{codegray},
	stringstyle=\color{blue},
	basicstyle=\ttfamily\footnotesize,
	breakatwhitespace=false,         
	breaklines=true,                 
	captionpos=b,                    
	keepspaces=true,                 
	numbers=left,                    
	numbersep=5pt,                  
	showspaces=false,                
	showstringspaces=false,
	showtabs=false,                  
	tabsize=2
}

%"mystyle" code listing set
\lstset{style=mystyle}


\AtBeginDocument{\addtocontents{toc}{\protect\thispagestyle{plain}}} 
\setcounter{tocdepth}{2}
%\linespread{1} 
\begin{document}
	
	%
	% Титульный лист на русском языке
	%
	% Название организации
	\institution{%		
		\begin{spacing}{1}
			Министерство науки и высшего образования Российской Федерации\\
			\textbf{--------------------------------}\\
			Санкт-Петербургский политехнический университет Петра Великого\\
			Высшая школа программной инженерии
		\end{spacing}
	}
	
	\title{Отчёт по курсовой работе}
	
	% Тема
	\topic{\textbf{"Методы имитационного моделирования"}}
	
	% Автор
	\author{Зайцев В.А.}
	\group{студент гр. 5130904/30103}
	
	% Научный руководитель
	\sa       {Дробинцев Д.Ф.}
	%\sastatus {проф, д.т.н. }
	
	% Город и год
	\city{Санкт-Петербург}
	\date{\number\year}
	
	\maketitle
	\pagestyle{plain}
	\newpage
	\tableofcontents
	
	\section{Постановка задачи}
	
	Целью курсовой работы является создание модели вычислительной системы (ВС) или ее части на некотором уровне детализации, описывающей и имитирующей ее структуру и функциональность.
	
	Каждый реальный объект (реальная ВС) обладает бесконечной сложностью, множеством характеристик, внутренних и внешних связей. Модель есть приближенное описание объекта с целью получения требуемых результатов с определенной точностью и достоверностью.
	
	При необходимости исследования поведенческих характеристик ВС в процессе исследования выгодно использовать не сам объект, а его модель. Степень приближения модели к описываемому объекту может быть различной и зависит от требований задачи.
	
	Существуют различные типы моделей:
	\begin{itemize}
		\item Аналитические (математические) модели
		\item Аналоговые модели
		\item Физические модели
		\item Имитационные модели
	\end{itemize}

	Последний тип моделей является предметом нашего изучения.
	
	Одним из подходов к построению имитационной модели является построение ее в виде системы массового обслуживания (СМО), с характерной для СМО терминологией: источник, буфер, прибор, диспетчер, заявка (требование).
	
	\section{Исходные данные}
	\begin{center}
		\fontsize{18}{15}\selectfont
		\textbf{Формализованная схема и описание СМО}
		
		\includegraphics[scale=0.5]{images/scheme.png}
	\end{center}

	\setlength\extrarowheight{3pt}
	\begin{table}[H]
		\centering
		\begin{tabular}{|l|l|l|l|l|l|l|l|l|l|}
			\hline
			11&ИБ&ИЗ1&ПЗ2&Д10З3&Д10О3&Д2П1&Д2Б2&ОР1&ОД2\\ \hline
		\end{tabular}
	\end{table}
	
	\begin{itemize}
		\item Параметры элементов модели.
	  	\begin{itemize}
		  	\item Источники: 
		  	\begin{itemize}
				\item ИБ~--- бесконечный источник;
				\item ИЗ1~--- пуассоновский закон распределения заявок;
		  	\end{itemize}
			\item Приборы: 	 
			\begin{itemize}
				\item ПЗ2~--- равномерный закон распределения времени обслуживания;
			\end{itemize}
		\end{itemize} 
		\item Описание дисциплин постановки и выбора.
		\begin{itemize}
			\item Буферизации: 
			\begin{itemize}
				\item Д1ОЗ3~--- постановка заявки в очередь на первое от начала свободное место;
			\end{itemize}
			\item Отказа: 	 
			\begin{itemize}
				\item Д1ОО3~--- заявка, раньше других вставшая в буфер получает отказ, уходит из системы и на её место встает пришедшая заявка;
			\end{itemize}
			\item Выбора заявки на обслуживание:
			\begin{itemize}
				\item Д2Б1~--- LIFO (последним пришел — первым обслужен);
			\end{itemize}
			\item Выбора обслуживающего прибора:
			\begin{itemize}
				\item Д2П1~--- приоритет по номеру прибора;
			\end{itemize}
		\end{itemize} 
		\item Виды отображения результатов работы программной модели.
		\begin{itemize}
			\item Динамическое отражение результатов (пошаговый режим): 
			\begin{itemize}
				\item ОД2~--- формализованная схема модели, текущее состояние;
			\end{itemize}
			\item Отражение	результатов	после сбора статистики (автоматический режим): 	 
			\begin{itemize}
				\item ОР1~--- сводная таблица результатов;
			\end{itemize}
		\end{itemize} 
	\end{itemize}
	\newpage
	
	\section{Пример временной диаграммы функционирования системы}
	\begin{center}
		
		\includegraphics[width=\textwidth]{images/waveform2.png}
	\end{center}
	
	\section{Обобщённая блок-схема}
	
	\begin{center}
		
		\includegraphics[width=\textwidth]{images/common_scheme.png}
	\end{center}
	
	\section{Модульная структура}
	Разработка производилась в среде VisualStudio на языке С++ и среде VS Code на языке Python с использованием графической библиотеки Qt.
	Приложение является объектно-ориентированным и содержит следующие классы: 
	\begin{itemize}
		\item Request~--- класс описывающий т пакет данных от датчика транспортного потока 
		\item Event~--- класс описывающий тип события и время его наступления и заявку, с которой это событие связано
		\item RequestSource~--- класс источника, который создаёт объекты заявок Request
		\item Handler~---  прибора, который обрабатывает заявки Request
		\item Buffer~--- класс буффера, хранящий некоторое число заявок Request
		\item UniformGenerator~--- класс генератора случайных чисел с равномерным распределением и в интервале $\left[a,b\right]$
		\item PoissonGenerator~--- класс генератора случайных чисел с экспоненциальным распределением с параметром $\lambda$
		\item QueueingSystem — класс реализующий интерфейс взаимодействия источников, приборов и буфера, а также сбор статистики 
		\item TimeDiagramPanel — визуализирует временные диаграммы работы системы 
		\item SimulationGUI — предоставляет интерфейс управления системой мониторинга
	\end{itemize}
	
	Примеры работы в пошаговом режиме
	
	\begin{center}
		\includegraphics[width=\textwidth]{images/step0.png}
	\end{center}
	\begin{center}
		\includegraphics[width=\textwidth]{images/step1.png}
	\end{center}
	\begin{center}
		\includegraphics[width=\textwidth]{images/step2.png}
	\end{center}
	\begin{center}
		\includegraphics[width=\textwidth]{images/step3.png}
	\end{center}
	
	Примеры работы в автоматическом режиме
	
	\begin{center}
		\includegraphics[width=\textwidth]{images/auto0.png}
	\end{center}

	\begin{center}
		\includegraphics[width=\textwidth]{images/auto1.png}
	\end{center}
	
	\section{Пример технической системы, удовлетворяющей формализованному описанию}
	
	\begin{table}[H]
		\centering
		\begin{tabular}{|p{5cm}|p{10cm}|}
			\hline
			Техническая система&Система видеонаблюдения\\ \hline
			Источники & Камеры видеонаблюдения, которые отправляют кадры только при обнаружении движения в кадре. Количество камер может варьироваться от 20 до 50 устройств.\\ \hline
			Приборы & Приборами являются видеокарты, которые анализируют кадры (поиск лиц, трекинг объектов между несколькими камерами).\\ \hline
			Буфер&Выделенная область оперативной памяти, в которую записываются поступающие кадры.\\ \hline
			Дисциплина постановки в	буфер&Постановка на первое свободное место в буфере\\ \hline
			Дисциплина выбора из буфера&LIFO - последние кадры обрабатываются первыми, так как они наиболее актуальны для текущего состояния системы.	\\ \hline
			Дисциплина отказа&Самые старые кадры в видеопамяти перезаписываются при нехватке места.\\ \hline
			Дисциплина постановки на обслуживание&Выбирается первая свободная видеокарта.\\ \hline
		\end{tabular}
	\end{table}
	
	\section{Ограничения и требуемые характеристики}	
	
	Вероятность отказа должна составлять не более 10\%.\\ 
	\indent Загрузка приборов более 90\%
	
	\begin{table}[H]
		\centering
		\begin{tabular}{|p{7cm}|p{5cm}|}
			\hline
			Количество камер & От 20 до 50\\ \hline
			Размер заявки & 1Мб\\ \hline
			Размер буфера & от 25Мб до 50Мб\\ \hline
			Количество видеокарт & от 1 до 10\\ \hline
			Среднее время между кадрами& $\lambda=1$с\\ \hline
			Время обработки кадров& 350-500 мс\\
					  			  &150-220 мс\\
								  &50-90 мс\\
								  &40-60 мс	\\ \hline
		\end{tabular}
	\end{table}

	\fontsize{18}{15}\selectfont
	\textbf{Стоимость компонентов системы}
	\normalsize
	\begin{table}[H]
		\centering
		\begin{tabular}{|p{6cm}|p{7cm}|p{3cm}|}
			\hline
			Наименование & Время обслуживания заявки & Цена\\ \hline
			NVIDIA GeForce GTX 1650 & 350-500 мс & 18 000 р.\\ \hline
			NVIDIA GeForce RTX 3050 & 150-220 мс & 40 000 р.\\ \hline
			NVIDIA GeForce RTX 4080 & 50-90 мс & 120 000 р.\\ \hline
			NVIDIA RTX A4000 & 40-60 мс & 150 000 р.\\ \hline
		\end{tabular}
	\end{table}

	Поскольку число камер и размер кадров таковы, что память буфера значительно меньше оперативной памяти стандартного компьютера (современный пк имеет от 4 Гб ОЗУ) более чем на порядок, то можем считать, что дополнительных затрат на память не будет. 
	
	Нам требуется найти самую дешёвую конфигурацию системы, которая будет удовлетворять всем требованиям при максимальной нагрузке т.е. при максимальном числе подключенных камер равном 50. Мы можем варьировать размер буфера и число камер, а также менять различные виды видеокарт. 	\\
	
	\section*{Результаты работы имитационной модели}
	\subsection*{Определение количества реализаций}
	\begin{equation*}
		N=\frac{t_{\alpha}^2(1-p)}{p\delta^2}=2429
	\end{equation*}
	\fontsize{18}{15}\selectfont
	\begin{center}
		\textbf{ NVIDIA GeForce GTX 1650}
	\end{center}
	\normalsize
	
	В случае с NVIDIA GeForce GTX 1650 мы можем отметить, что на всех конфигурациях она выдаёт максимально возможную или близкую к максимальной загрузку всех видеокарт, но при этом ни одна конфигурация не позволяет снизить вероятность отказа хотя бы до 50\% не говоря уже о целевых 10\%.
	
	\begin{figure}[H]
		\centering
		\includegraphics[width=\textwidth]{images/1650_usage.pdf}
	\end{figure}

	\begin{figure}[H]
		\centering
		\includegraphics[width=\textwidth]{images/1650_rejection_rate.pdf}
	\end{figure}

	\begin{figure}[H]
		\centering
		\includegraphics[width=\textwidth]{images/1650_total.pdf}
	\end{figure}
	
	
	\fontsize{18}{15}\selectfont
	\begin{center}
		\textbf{NVIDIA GeForce RTX 3050}
	\end{center}
	\normalsize
	
	Для NVIDIA GeForce RTX 3050 можно найти несколько конфигураций которые будут удовлетворять всем необходимым условиям. Первый набор конфигураций состоит из 9 видеокарт и буфера размером от 5 до 25 заявок. Второй набор состоит из 10 видеокарт буфера размером от 8 до 25 заявок. Самой оптимальной будет конфигурация из 9 видеокарт и 5 местного буфера, однако при необходимости можно снизить вероятность отказа за счёт  
	
	\begin{figure}[H]
		\centering
		\includegraphics[width=\textwidth]{images/3050_usage.pdf}
	\end{figure}
	
	\begin{figure}[H]
		\centering
		\includegraphics[width=\textwidth]{images/3050_rejection_rate.pdf}
	\end{figure}
	
	\begin{figure}[H]
		\centering
		\includegraphics[width=\textwidth]{images/3050_total.pdf}
	\end{figure}
	
	\fontsize{18}{15}\selectfont
	\begin{center}
		\textbf{NVIDIA GeForce RTX 4080}
	\end{center}
	\normalsize
	
	
	\begin{figure}[H]
		\centering
		\includegraphics[width=\textwidth]{images/4080_usage.pdf}
	\end{figure}
	
	\begin{figure}[H]
		\centering
		\includegraphics[width=\textwidth]{images/4080_rejection_rate.pdf}
	\end{figure}
	
	\begin{figure}[H]
		\centering
		\includegraphics[width=\textwidth]{images/4080_total.pdf}
	\end{figure}
	
	\fontsize{18}{15}\selectfont
	\begin{center}
		\textbf{NVIDIA RTX A4000}
	\end{center}
	\normalsize
		
	\begin{figure}[H]
		\centering
		\includegraphics[width=\textwidth]{images/A400_usage.pdf}
	\end{figure}
	
	\begin{figure}[H]
		\centering
		\includegraphics[width=\textwidth]{images/A400_rejection_rate.pdf}
	\end{figure}
	
	\begin{figure}[H]
		\centering
		\includegraphics[width=\textwidth]{images/A400_total.pdf}
	\end{figure}
	
	Таким образом мы получили несколько вариантов конфигурации удовлетворяющих ограничениям. Все подходящие конфигурации используют NVIDIA GeForce RTX 3050. Самой оптимальной из подходящих является вариант с 9 видеокартами и 5 местами в буфере, однако можно добавить ещё одну видеокарту или увеличить число мест в буфере до максимальных 25 чтобы ещё сильнее уменьшить вероятность отказа оставаясь в рамках необходимой загрузки. Остальные видеокарты в каждой конфигурации не могут удовлетворить хотя бы одному параметру. 
	
	\section{Вывод}
	В ходе курсовой работы была написана имитационная модель системы массового обслуживания на языке C++, а также реализован графический интерфейс для неё на языке Python с использованием графической библиотеки Qt. С помощью данной программы была проанализирована реальная система видеонаблюдения и подобрана максимально выгодная комплектация системы, удовлетворяющая поставленным требованиям.
	
	\section{Список литературы}
	
\end{document}